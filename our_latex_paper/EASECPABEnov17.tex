\documentclass[conference]{IEEEtran}
\IEEEoverridecommandlockouts
% The preceding line is only needed to identify funding in the first footnote. If that is unneeded, please comment it out.
\usepackage{cite}
\usepackage{amsmath,amssymb,amsfonts,nccmath}
\usepackage{algorithmic}
\usepackage{graphicx}
\usepackage{textcomp}
\usepackage{xcolor}
\usepackage{comment}
\usepackage{lipsum}% http://ctan.org/pkg/lipsum
\def\BibTeX{{\rm B\kern-.05em{\sc i\kern-.025em b}\kern-.08em
T\kern-.1667em\lower.7ex\hbox{E}\kern-.125emX}}
\begin{document}


\section{\textbf{PROPOSED ECC-CP-ABE SCHEME}}

We propose a novel ECC-based CP-ABE scheme known as Ecc-bAsed ScalablE Revocation (EASER) CP-ABE. The scheme comprises of a trusted server which performs the setup and key distribution. It also used a trusted proxy server that maintains a Revocation List of revoked users and generates proxy components to assist in partial decryption.
\begin{itemize}
\item{\textbf{Trusted server }}
The trusted server of the system performs the setup, key generation and encryption of the data. It retains details for registered users, such as user identification, name and also issues the user's secret keys.

\item{\textbf{Proxy Server }}
It maintains a revocation list of revoked user identities.
It also maintains partial secret keys of all users as discussed in the keygen step. It sends proxy component Q to the user to assist in partial decryption, such that if a user is present in the revocation list, the decryption fails for only the revoked users. The other valid users can decrypt uninterruptedly without any requirement of re-encryption or re-distribution of keys.
\end{itemize}

\subsection{\textbf{Phases of the EASER CPABE scheme:}}
The scheme comprises the following phases: Setup, Encryption, KeyReg and Decryption.

%setup=======================
%setup=======================

\subsection{\textbf{Setup phase}}
%{\color{red}{Mark all variables in italics}}\\
The setup phase is the initialisation phase. The size $n$ of the Universe of attributes $U$ identifies how many different attributes are present in the user secret keys. A bit $0$ or $1$ denotes an attribute is present or absent.
For $n =4$, there are $4$ attributes in a user’s key and system policy that encrypts the data. A sample attribute definition can be: \{\emph{‘IT Professional?’}, \emph{‘joined before march 2016?’}, \emph{‘In an ongoing project?’}, \emph{‘Access to top secret files?’}\}. A sample attribute set can be $\mathbb{A}   = \{0 0 1 0\}$ In the same way a sample system policy for encryption can be $\mathbb{P}  = \{0 0 1 0\}$. In this case, since the attributes associated with user's secret key satisfy the system's access policy, the user can successfully decrypt the ciphertext using the secret keys denoted by attribute set $\mathbb{A}$. If the Policy is $\mathbb{P} = \{0 0 1 0\}$, then the user will not be able to decrypt, since the attribute set A does not satisfy the policy $\mathbb{P}$.

\begin{itemize}
\item{\bf SP1:} Choose the parameters for an Elliptic Curve group $\mathbb{G} = \{p, E\textsubscript{p}(a, b), P\}$, where $P$ is the base point on the curve and $p$ is some large prime number which defines the field $Z\textsubscript{p}$ of the curve. The order of the curve generated must be prime in order to allow ECC point division. It is best to follow the standard NIST ECC curves which provide the best security and also adhere to all requirements of the EASER scheme.

\item{\bf SP2} : Within the finite field of $p$ generate three random numbers $\alpha$, $k$\textsubscript{1} and $k$\textsubscript{2} such that $\{\alpha, k\textsubscript{1}, k\textsubscript{2}\} \neq0$. Calculate the values of $P\textsubscript{i}$, $U\textsubscript{i}$ and $V\textsubscript{i}$ $\forall i \in 0,1\ldots  n$ as follows:
\begin{ceqn}
\begin{align}
P\textsubscript{i} &= \alpha\textsubscript{i}P \\
U\textsubscript{i} &= k\textsubscript{1}\alpha\textsuperscript{i}P \\
V\textsubscript{i} &= k\textsubscript{2}\alpha\textsuperscript{i}P
\end{align}
\end{ceqn}

\item{\bf SP3} : Choose any $4$ trapdoor hash functions $H\textsubscript{1}$, $H\textsubscript{2}$, $H\textsubscript{3}$ and $H\textsubscript{4}$ defined as follows :
\begin{ceqn}
\begin{align}
H_1,H_4 \quad&: \{0, 1\}\textsuperscript{*}\longrightarrow Z\textsubscript{p}\textsuperscript{*} \\
H_2 \quad&: \{0, 1\}\textsuperscript{*}\longrightarrow \{0, 1\}\textsuperscript{$\mid$$\sigma$$\mid$} \\
H_3 \quad&: \{0, 1\}\textsuperscript{*}\longrightarrow \{0, 1\}\textsuperscript{$\mid$m$\mid$}
\end{align}
\end{ceqn}

where $\sigma$ is some large string generated randomly, $M$ is the message to be encrypted and $\mid . \mid$ is the operator to find the length of a string within it and $\{0,1\}\textsuperscript{p}$ denotes a random stream of binary digits of length $p$ and $*$ denotes any number from $0 \longrightarrow \infty $.
\item{\textbf{SP4:}} Using the above valuesm it generates the the global secret key \emph{GSK} and global public key \emph{GPK} $ \forall i \in 0,1,\ldots , n$ as follows:
\begin{ceqn}
\begin{align}
GSK &= \{ \alpha, k\textsubscript{1},k\textsubscript{2}\}\\
GPK &= \{ \mathbb{G}, P\textsubscript{i}, V\textsubscript{i}, U\textsubscript{i}, H\textsubscript{1}, H\textsubscript{2}, H\textsubscript{3}, H\textsubscript{4} \}
\end{align}
\end{ceqn}
\end{itemize}

{\color{red} {To remove these:}}
k{\textsubscript{1}



%==============keygen phase==================================
%==============keygen phase==================================
%==============keygen phase==================================


\subsection{\textbf {KeyReg Phase}}
This phase assigns User-ID(\emph{UID\textsubscript{j}}) to user $j$. It takes as input the \emph{GSK}, the \emph{GPK} and the Employee-ID(\emph{EID\textsubscript{j}}) or credentials of a user to generate a user secret key $k\textsubscript{u}$.
% \begin{itemize}
% \item{\color{red}Change the below as items and equations as in the setup}
% \end{itemize}
\begin{itemize}
\item{\textbf{KP1:}} Let $\mathbb{A}  = \{a\textsubscript{1}, a\textsubscript{2}, \ldots, a\textsubscript{n}\}$ be the attribute set for the user. Compute the value of the expression:

\begin{ceqn}
    \begin{align}
    f(\alpha , \mathbb{A} ) &= \prod_{i=1}^{n}(\alpha + H_4(i))^{(1-i)} 
    \end{align}
\end{ceqn}
    
The polynomial function $f(x, \mathbb{A} )$ is of degree at-most $n$.

\item{\textbf {KP2}} : Within the finite field $Z\textsubscript{p}$, generate two random numbers $r\textsubscript{u}$ and $t\textsubscript{u}$, and compute the following values: 
\begin{ceqn}
    \begin{align}
        u_1 &= r_u+k_1.f(x, \mathbb{A} )\\
        u_2 &= s_u - k_2.f(x, \mathbb{A} )\\
        u_3 &= r_u k_2 + t_u  k_1
    \end{align}
\end{ceqn}
    
Send the portion of the keys $u\textsubscript{1}$ and $u\textsubscript{2}$ to the user and $u\textsubscript{3}$ to the proxy server.

\item{\textbf {KP3}} : Sends the portion of the secret key $u\textsubscript{3} .  f(x, \mathbb{A} )$ to the proxy server.
It provides the following identity and portion of the secret keys to the user \emph{j}:

\begin{ceqn}
    \begin{align}
        k_u &= \{uid_j, u_1, u_2\}
    \end{align}
\end{ceqn}

\end{itemize}

%==============Encryption==================================
%==============Encryption==================================
%==============Encryption==================================

\subsection{\textbf {Encryption Phase: }}
It takes the plaintext, access policy and GPK as input. Standard AES algorithm encrypt the plaintext and the EASER CPABE scheme encrypts the AES key.
% {\color{red}{itemize the steps and put equations}}\\

\begin{itemize}
    
\item{\textbf {EP1:}} Generate a random number $\sigma$. Now generate another random number $\sigma$\textsubscript{m} $\in$ \{0,1\}\textsuperscript{$\mid\sigma\mid$}. 

{\color{red}{What is the difference between the two?}}

\item{\textbf{EP2:}} Compute the values of $r\textsubscript{m}$ and $k\textsubscript{m}$ according to the expression: 
\begin{ceqn}
    \begin{align}
        r_m &= H_1(\mathbb{P}, M, \sigma_m )\\
        k_m &= KDF(r_m\mathbb{P} )
    \end{align}
\end{ceqn}

\item{\textbf {EP3:}} For the access policy $\mathbb{P} = \{b\textsubscript{1}, b\textsubscript{2}, \ldots, b\textsubscript{n}\}$ , compute the following:

\begin{ceqn}
    \begin{align}
        f(x,\mathbb{P} ) &= \prod_{i=1}^n(x+H_4(i))^{(1-b_i)} 
    \end{align}
\end{ceqn}

The polynomial function $\ f(x, \mathbb{P} )$ is of degree at-most $n$. Let $f\textsubscript{i}$ denote the coefficient of $x\textsubscript{i}$ in the polynomial function $f(x,\mathbb{P} )$ .

\item{\textbf {EP4:}} Compute the following: 
\begin{ceqn}
    \begin{align}
        P_{m,i} &= r_m P_i , i = 1, \ldots, n - |\mathbb{P} |\\
        K_{1,m} &= r_m \sum_{i=0}^n f_i U_i \\ 
        K_{2,m} &= r_m \sum_{i=0}^n f_i V_i 
    \end{align}
    \begin{align}
        C_{\sigma_m} &= H_2(k_m) \oplus \sigma_m \\ 
        C_m &= H_3(\sigma_m) \oplus M 
    \end{align}
\end{ceqn}


{\color{red} Do not put like this expand equations and put properly in the above:}\\

It can be conveniently proved that K\textsubscript{1m} and K\textsubscript{2m} results into r\textsubscript{m}k\textsubscript{1} $\ f$($\alpha$,$\mathbb{P}$)P and r\textsubscript{m}k\textsubscript{2} $\ f$($\alpha$,$\mathbb{P}$)P respectively, by putting the values of U\textsubscript{i} and V\textsubscript{i} in the respective expressions.\\



%\noindent{\bf Phase 5} : In a new file, output the ciphertext as C where \\

%$\-$ $\-$ $\-$ $\-$ $\-$ $\-$ $\-$ $\-$ $\-$$\-$$\-$$\-$ C = \{AES ciphertext, $\mathbb{P}$,P\textsubscript{m,i},k\textsubscript{1m},k\textsubscript{2m},C,C\textsubscript{m}\}
\end{itemize}


%==============decryption==================================
%==============decryption==================================
%==============decryption==================================

\subsection{\textbf {Decryption Phase: }}

The decryption phase takes as input Ciphertext along with user's secret key k\textsubscript{u} = \{uid\textsubscript{j}, u\textsubscript{1}, u\textsubscript{2}\} and generates the plaintext as as output. \\
{\color{red}{Change the following to itemize and equations}} \\
\noindent{\bf DP1:} If access policy P is not a subset of attribute set A, then abort.\\

\noindent{\bf DP2} : Compute the values U and V as follows:\\

U = u\textsubscript{2}k\textsubscript{1m} = (t\textsubscript{u}-k\textsubscript{2}$\ f(\alpha,\mathbb{A})$)(r\textsubscript{m}k\textsubscript{1}$\ f(\alpha,\mathbb{P})P$)\\

V = u\textsubscript{1}k\textsubscript{2m} = (r\textsubscript{u}-k\textsubscript{1}$\ f(\alpha,\mathbb{A})$)(r\textsubscript{m}k\textsubscript{2}$\ f(\alpha,\mathbb{P})P$)\\

Send U, V and user id uid\textsubscript{j} to the proxy server. The proxy server sends proxy component Q back to the user to assist partial decryption and scalable revocation. \\
%=========================================
%=========================================
%=========================================

\par \textbf {Proxy Server }
\par It maintains a revocation list which comprises of user ids that must revoked users. The proxy server generates proxy components and sends them to all users. For revoked users, it modifies the proxy components so that decryption fails. The other valid users get proper proxy components so that description is successful and they access the ciphertext without an interruption.

\par A user send uid\textsubscript{j}, U and V to the proxy server.
The proxy server also maintains a list of user ids as well as the secret key components (u\textsubscript{3}, f($\alpha$,$\mathbb{A}$)) as discussed in the KeyGen phase.
The proxy server does the following to generate the proxy components: \\
\noindent{\bf DP2.1} :
R = U+V \\
Therefore,\\

R = U+V = $(t\textsubscript{u}-k\textsubscript{2}\ f(\alpha,\mathbb{A}))(r\textsubscript{m}k\textsubscript{1}\ f(\alpha,\mathbb{P})P)+(r\textsubscript{u}-k\textsubscript{1}\ f(\alpha,\mathbb{A}))(r\textsubscript{m}k\textsubscript{2}\ f(\alpha,\mathbb{P})P$) \\

= t\textsubscript{u}r\textsubscript{m}k\textsubscript{1}$\ f(\alpha,\mathbb{P})$ - r\textsubscript{m}k\textsubscript{1}k\textsubscript{2}$\ f(\alpha,\mathbb{P})$$\ f(\alpha,\mathbb{A})P$ \\

+r\textsubscript{u}r\textsubscript{m}k\textsubscript{2}$\ f(\alpha,\mathbb{P})$ - r\textsubscript{m}k\textsubscript{1}k\textsubscript{2}$\ f(\alpha,\mathbb{P})$$\ f(\alpha,\mathbb{A})P$ \\

= r\textsubscript{m}(r\textsubscript{u}k\textsubscript{2} + t\textsubscript{u}k\textsubscript{1})}$\ f(\alpha,\mathbb{P})P$\\

= r\textsubscript{m}u\textsubscript{3}$\ f(\alpha,\mathbb{P})P$\\

%Hence, R = r\textsubscript{m}u\textsubscript{3}$\ f$($\alpha$,$\mathbb{P}$)P \\


\noindent{\bf DP2.2} : The proxy server checks if the user id uid\textsubscript{3} is registered and also present in the revocation list.\\

\textbf{Case No Revocation} : For a valid user id calculate proxy component Q as: \\

$\-$ $\-$ $\-$ $\-$ $\-$ $\-$ $\-$ $\-$ $\-$$\-$$\-$$\-$ Q = $\frac{R}{X\textsubscript{i}}$ \\

where X\textsubscript{i} is the value mapped to UID in the memory.\\

Since, X\textsubscript{i}= u\textsubscript{3}$\ f$($\alpha$,$\mathbb{A}$) \\

Q = $\frac{R}{X\textsubscript{i}}$ = $\frac{U+V}{\ f(\alpha,\mathbb{A})}$ = $\frac{r\textsubscript{m}u\textsubscript{3}\ f(\alpha,\mathbb{P})P}{u\textsubscript{3}\ f(\alpha,\mathbb{A})}$\\

= r\textsubscript{m}$\ F$($\alpha$,$\mathbb{P}$)P \\

where $\digamma$($\alpha$) = $\frac{\ f(\alpha,\mathbb{P})}{\ f(\alpha,\mathbb{A})}$ \\
The proxy server returns the proxy component Q to the valid user. \\


\textbf{Case Revocation}: If user id uid\textsubscript{j} is present in the revocation list, then compute the proxy component Q as follows: \\

Q = $\frac{R}{B}$, where B is some random number such that Q is not same as r\textsubscript{m}$\ F$($\alpha$,$\mathbb{P}$)P. Return Q to the user so that decryption will fail.

\noindent{\bf DP3:} Evaluate the expression for $\digamma$(x) where $\digamma$(x) = $\digamma$(x,$\mathbb{A}$,$\mathbb{P}$) is defined as \\

$\-$ $\-$ $\-$ $\-$ $\-$ $\-$ $\-$ $\-$ $\-$$\-$$\-$$\-$ $\digamma$(x,$\mathbb{A}$,$\mathbb{P}$) = $\sum_{i = 1}^{n-\mid\mathbb{P}\mid}$ (x+H\textsubscript{4}(i))\textsuperscript{c\textsubscript{i}} \\

where c\textsubscript{i}=a\textsubscript{i} - b\textsubscript{i}

\noindent Let $\digamma$\textsubscript{i} be the coefficient of x\textsubscript{i} in $\digamma$(x). Since $\mathbb{P}$ $\subseteq$ $\mathbb{A}$,\\
$\digamma$\textsubscript{0} $\geq$ 1 .\\

\noindent{\bf DP4: } Calculate the value of W using the expression \\

$\-$ $\-$ $\-$ $\-$ $\-$ $\-$ $\-$ $\-$ $\-$$\-$$\-$$\-$ W= $\sum_{i = 1}^{n-\mid \mathbb{P} \mid}$ $\digamma$\textsubscript{i}P\textsubscript{m,i} \\

$\-$ $\-$ $\-$ $\-$ $\-$ $\-$ $\-$ $\-$ $\-$$\-$$\-$$\-$ = r\textsubscript{m}($\sum_{i = 1}^{n-\mid \mathbb{P} \mid} (\digamma\textsubscript{i}\alpha\textsuperscript{i})P$ \\

$\-$ $\-$ $\-$ $\-$ $\-$ $\-$ $\-$ $\-$ $\-$$\-$$\-$$\-$ = r\textsubscript{m}($\sum_{i = 1}^{n-\mid \mathbb{P} \mid} (\digamma\textsubscript{i}\alpha\textsuperscript{i} + \digamma\textsubscript{0}-\digamma\textsubscript{0})P$ \\

$\-$ $\-$ $\-$ $\-$ $\-$ $\-$ $\-$ $\-$ $\-$$\-$$\-$$\-$ = r\textsubscript{m}($(\digamma)$\textsubscript{($\alpha$)} - $\digamma$\textsubscript{0})P\\

$\-$ $\-$ $\-$ $\-$ $\-$ $\-$ $\-$ $\-$ $\-$$\-$$\-$$\-$ = r\textsubscript{m}$\digamma$\textsubscript{$\alpha$}P - r\textsubscript{m}$\digamma$\textsubscript{0}P \\

Compute $\frac {1}{\digamma\textsubscript{0}}$(Q-W).

For a valid user: \\

$\frac {1}{\digamma\textsubscript{0}}$(Q - W) = $\frac {1}{\digamma\textsubscript{0}}$(r\textsubscript{m}F$\alpha$P - (r\textsubscript{m}$\digamma(\alpha)$ - r\textsubscript{m}$\digamma$\textsubscript{0}))P \\

%Since Q = r\textsubscript{m} $\digamma$($\alpha$)P and W =r\textsubscript{m}$\digamma(\alpha)$ - r\textsubscript{m}$\digamma$\textsubscript{0}P (from phase 4 of decryption)\\

= $\frac {1} {\digamma\textsubscript{0}}$(r\textsubscript{m}$\digamma$\textsubscript{0}P)\\

= r\textsubscript{m}P\\
\par For a revoked user the above computation fails : \\
$\frac {1}{\digamma\textsubscript{0}}$(Q - W) != r\textsubscript{m}P\\

\noindent{\bf DP5:} Compute :\\

$\-$ $\-$ $\-$ $\-$ $\-$ $\-$ $\-$ $\-$ $\-$$\-$$\-$$\-$ $\sigma$'\textsubscript{m}=H\textsubscript{2}(KDF(r\textsubscript{m}P)) $\oplus$ C\textsubscript{$\sigma$\textsubscript{m}}\\

$\-$ $\-$ $\-$ $\-$ $\-$ $\-$ $\-$ $\-$ $\-$$\-$$\-$$\-$ M' = C\textsubscript{m} $\oplus $H\textsubscript{3}($\sigma$'\textsubscript{m}) \\

$\-$ $\-$ $\-$ $\-$ $\-$ $\-$ $\-$ $\-$ $\-$$\-$$\-$$\-$ r'\textsubscript{m} = H\textsubscript{1}($\mathbb{P}$,M',$\sigma$'\textsubscript{m}) \\

$\-$ $\-$ $\-$ $\-$ $\-$ $\-$ $\-$ $\-$ $\-$$\-$$\-$$\-$ Verify if r’\textsubscript{m}P = r\textsubscript{m}P in phase DP4, then the message M is valid.

%=========================================
%=========================================
%=========================================

\subsection{Fix for Attack}
\textbf {Theorem 1: } Proposed EASER CP-ABE scheme is secure and does not allows deriving of system private keys (k\textsubscript{1},k\textsubscript{2}) by collusion attack. {\color{red}referemce}\\

Proof: u\textsubscript{1}\textsuperscript{i} = r\textsubscript{u\textsubscript{i}} + k\textsubscript{1}$\ f$($\alpha$,$\mathbb{A}$)

$\-$ $\-$ $\-$ $\-$ $\-$ $\-$ $\-$ $\-$ $\-$ $\-$ $\-$$\-$$\-$$\-$ u\textsubscript{2}\textsuperscript{i} = t\textsubscript{u\textsubscript{i}} + k\textsubscript{1}$\ f$($\alpha$,$\mathbb{A}$) \hspace{10mm} for i=1,......l

Here the system is of '2l' linear equations and 2l+2 unknowns. Hence not solvable.
\end{document}
